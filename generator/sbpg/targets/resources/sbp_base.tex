\documentclass[9pt]{extarticle}

\usepackage{amsmath}
\usepackage{booktabs}
\usepackage{bytefield}
\usepackage{caption}
\usepackage{endnotes}
\usepackage{fancyvrb}
\usepackage{float}
\usepackage{longtable}
\usepackage{minibox}
\usepackage{register}
\usepackage{standalone}
\usepackage{swiftnav}
\usepackage{tabularx}
\usepackage{tocloft}
\usepackage{setspace}
\usepackage{pbox}
\usepackage{soul}
\usepackage{hyperref}
\usepackage{ltxtable}
\usepackage{caption}

\hypersetup{bookmarks,bookmarksopen,bookmarksdepth=4}

\setlength{\regWidth}{0.4\textwidth}

\floatstyle{plain}
\newfloat{field}{h}{fld}
\floatname{field}{Field}

\numberwithin{table}{subsection}
\numberwithin{field}{subsection}

\renewcommand{\version}{(((version)))}
\renewcommand{\thesubsubsection}{\hspace{-0.45cm}}

\newcommand{\specialcell}[2][c]{%
  \begin{tabular}[#1]{@{}c@{}}#2\end{tabular}}

\renewcommand{\regLabelFamily}{}

\cftsetindents{section}{0.5in}{0.5in}
\cftsetindents{subsection}{0.5in}{0.5in}
%%\setlength\cftparskip{-1.2pt}
\setlength{\cftbeforetoctitleskip}{-1em}
\setlength\cftbeforesecskip{1.3pt}
\setlength\cftaftertoctitleskip{2pt}
\renewcommand{\cftsecafterpnum}{\hspace*{7.5em}}
\renewcommand{\cftsubsecafterpnum}{\hspace*{7.5em}}
\renewcommand\tableofcontents{\@starttoc{toc}}

\newcolumntype{a}{>{\hsize=.2\hsize}X}
\newcolumntype{b}{>{\hsize=.22\hsize}X}
\newcolumntype{c}{>{\hsize=.3\hsize}X}
\newcolumntype{d}{>{\hsize=.7\hsize}X}
\newcolumntype{e}{>{\hsize=.13\hsize}X}
\newcolumntype{f}{>{\hsize=.16\hsize}X}
\newcolumntype{g}{>{\hsize=.77\hsize}X}
\newcolumntype{h}{>{\hsize=.6\hsize}X}

% Shell out to git to get the most recent tag and pass it to the LateX
% job name. Hopefully this doesn't screw with things.
\immediate\write18{git describe --abbrev=0 --tags | cut -c 2-5 > \jobname.info }

\title{Swift Navigation Binary Protocol}
\author{Swift Navigation}
\mysubtitle{Protocol Specification \version}
\date{\today}

\begin{document}
\maketitle
\begin{normalsize}
\setcounter{tocdepth}{2}
\begin{centering}
\tableofcontents
\end{centering}
\end{normalsize}

\thispagestyle{firstpage}
\bigskip
\bigskip
\begin{large}
\section{Overview}
\label{sec:Overview}
The Swift Navigation Binary Protocol (SBP) is a fast, simple, and
minimal binary protocol for communicating with Swift devices. It is
the native binary protocol used by the Piksi GPS receiver to transmit
solutions, observations, status, and debugging messages, as well as
receive messages from the host operating system, such as differential
corrections and the almanac. As such, it is an important interface
with your Piksi receiver and the primary integration method with other
systems.

This document provides a specification of SBP framing and the payload
structures of the messages currently used with Swift devices. SBP
client libraries in a variety of programming languages are available
at~\url{https://github.com/swift-nav/libsbp} and support information
for sbp is available at~\url{https://support.swiftnav.com/customer/en/portal/articles/2492810-swift-binary-protocol}.

\end{large}

\newpage
\section{Message Framing Structure}
\label{sec:Message}

\begin{large}
SBP consists of two pieces:
\begin{itemize}
  \item an over-the-wire message framing format
  \item structured payload definitions
\end{itemize}
As of Version~\version, the frame consists of a 6-byte binary
header section, a variable-sized payload field, and a 16-bit CRC
value. All multibyte values are ordered in \textbf{little-endian}
format. SBP uses the CCITT CRC16 (XMODEM implementation) for error
detection\footnote{CCITT 16-bit CRC Implementation uses parameters
  used by XMODEM, i.e. the polynomial: $x^{16} + x^{12} + x^5 +
  1$. For more details, please see the implementation
  at~\url{https://github.com/swift-nav/libsbp/blob/master/c/src/edc.c\#L59}. See
  also \emph{A Painless Guide to CRC Error Detection Algorithms}
  at~\url{http://www.ross.net/crc/download/crc_v3.txt}}.

\end{large}

\begin{table}[h]
  \centering
  \begin{tabularx}{\textwidth}{aaaX}
    \toprule
    Offset (bytes) & Size (bytes) & Name & Description \\
    \midrule
    $\mathtt{0}$ & $\mathtt{1}$ & {Preamble} & Denotes the start of frame transmission. Always 0x55. \\
    $\mathtt{1}$ & $\mathtt{2}$ & {Message Type} & Identifies the payload contents. \\
    $\mathtt{3}$ & $\mathtt{2}$ & {Sender} & A unique identifier of the sender.\footnotemark \\
    $\mathtt{5}$ & $\mathtt{1}$ & {Length} & Length (bytes) of the {Payload} field. \\
    $\mathtt{6}$ & $\mathtt{N}$ & {Payload} & Binary message contents. \\
    $\mathtt{N+6}$ & $\mathtt{2}$ & {CRC} & \hangindent=0.5em{Cyclic Redundancy Check of the frame's binary data from the Message Type up to the end of Payload (does not include the Preamble).} \\
    \midrule
    & $\mathtt{N+8}$ & &Total Frame Length \\
    \bottomrule
  \end{tabularx}
  \caption{Swift Binary Protocol message structure. $\mathtt{N}$ denotes a variable-length size.}
  \label{tab:message}
\end{table}
\footnotetext{By default, clients of `libsbp` use a sender id value of `0x42` which represents device controllers such as the Piksi Console. On the Piksi, the sender ID is set to the 2 least significant bytes of the device serial number. A stream of SBP messages may also include sender IDs for forwarded messages from other systems. For instance, when using Starling as a hosted software product, Sender 0x1000 (4096) indicates a message originated from the GNSS subsystem, while sender 0x315 (789) indicates a message originated from the sensor fusion subsystem. Sender 0 always indicates the message has been forwarded and contains some form of differential corrections.}

\section{NMEA-0183}
\label{sec:NMEA}

\begin{large}

Swift devices, such as the Piksi, also have limited support for the standard
NMEA-0183 protocol.

Note that NMEA-0183 doesn't define standardized message
string equivalents for many important SBP messages such as observations,
baselines and ephemerides. For this reason it is strongly recommended to use
SBP for new development. NMEA-0183 output is provided primarily to support
legacy devices.

\end{large}

\newpage

\section{Basic Formats and Payload Structure}
\label{sec:Payload}
\begin{large}
The binary payload of an SBP message decodes into structured data
based on the message type defined in the header. SBP uses several
primitive numerical and collection types for defining payload
contents.
\end{large}
\begin{table}[h]
  \centering
  \begin{tabularx}{\textwidth}{aaX}
    \toprule
    Name & Size (bytes) & Description \\
    \midrule
    ((*- for t in prims *))
    (((t.identifier))) & (((t.identifier | getsize))) & \hangindent=0.5em{(((t.desc)))} \\
    ((*- endfor *))
    \bottomrule
  \end{tabularx}
  \caption{SBP primitive types}
  \label{tab:types}
\end{table}
\hspace{-5em}
\subsubsection*{Example Message}
\begin{large}
 \par As an example, consider this framed series of bytes read from a
 serial port:
\begin{verbatim}
55 0b 02 cc 04 14 70 3d d0 18 cf ef ff ff ef e8 ff ff f0 18 00 00 00 00 05 00 15 dc
\end{verbatim}
This byte array decodes into a \texttt{MSG\_BASELINE\_ECEF} (see
pg.~\pageref{sec:MSG_BASELINE_ECEF}), which reports the baseline position
solution of the rover receiver relative to the base station receiver
in Earth Centered Earth Fixed (ECEF) coordinates. The segments of this
byte array and its contents break down as follows:
\end{large}
\begin{table}[h]
  \centering
  \begin{tabular}{llrl}
    \toprule
    Field Name & Type & Value & Bytestring Segment\\
    \midrule
    {Preamble} & u8 & 0x55 & \verb!55! \\
    {Message Type}& u16 & \texttt{MSG\_BASELINE\_ECEF} & \verb!0b 02! \\
    {Sender}& u16 & 1228 & \verb!cc 04! \\
    {Length}& u8 & 20 &  \verb!14! \\
    {Payload}& & --- & \verb!70 3d d0 18 cf ef ff ff ef e8 ff ff! \\
    & & & \verb!f0 18 00 00 00 00 05 00! \\
    \quad~\texttt{MSG\_BASELINE\_ECEF} & & & \\
    \quad~.tow & u32 & $416300400~\textrm{msec}$  & \verb!70 3d d0 18! \\
    \quad~.x & s32 & $-4145~\textrm{mm}$  & \verb!cf ef ff ff! \\
    \quad~.y & s32 & $-5905~\textrm{mm}$  & \verb!ef e8 ff ff! \\
    \quad~.z & s32 & $6384~\textrm{mm}$  & \verb!f0 18 00 00! \\
    \quad~.accuracy & u16 & 0 & \verb!00 00! \\
    \quad~.nsats & u8 & 5 & \verb!05! \\
    \quad~.flags & u8 & 0 & \verb!00! \\
    {CRC} & u16 & 0x9443 & \verb!15 dc! \\
    \bottomrule
  \end{tabular}
  \caption{SBP breakdown for \texttt{MSG\_BASELINE\_ECEF}}
  \label{tab:example_breakdown}
\end{table}

\newpage

\section{Gnss Signals}
\label{sec:signals}
\begin{large}
Code, Constellation, and Band
\end{large}\\

\begin{center}
  \begin{longtable}{{llllp{6.5cm}}}
    \toprule
    Value & Description & Value & Description \\
    \midrule
    {0} & {GPS L1CA} & {33} & {QZS L1CQ} \\
    {1} & {GPS L2CM} & {34} & {QZS L1CX} \\
    {2} & {SBAS L1CA} & {35} & {QZS L2CM} \\
    {3} & {GLO L1OF} & {36} & {QZS L2CL} \\
    {4} & {GLO L2OF} & {37} & {QZS L2CX} \\
    {5} & {GPS L1P} & {38} & {QZS L5I} \\
    {6} & {GPS L2P} & {39} & {QZS L5Q} \\
    {7} & {GPS L2CL} & {40} & {QZS L5X} \\
    {8} & {GPS L2CX} & {41} & {SBAS L5I} \\
    {9} & {GPS L5I} & {42} & {SBAS L5Q} \\
    {10} & {GPS L5Q} & {43} & {SBAS L5X} \\
    {11} & {GPS L5X} & {44} & {BDS3 B1CI} \\
    {12} & {BDS2 B1} & {45} & {BDS3 B1CQ} \\
    {13} & {BDS2 B2} & {46} & {BDS3 B1CX} \\
    {14} & {GAL E1B} & {47} & {BDS3 B5I} \\
    {15} & {GAL E1C} & {48} & {BDS3 B5Q} \\
    {16} & {GAL E1X} & {49} & {BDS3 B5X} \\
    {17} & {GAL E6B} & {50} & {BDS3 B7I} \\
    {18} & {GAL E6C} & {51} & {BDS3 B7Q} \\
    {19} & {GAL E6X} & {52} & {BDS3 B7X} \\
    {20} & {GAL E7I} & {53} & {BDS3 B3I} \\
    {21} & {GAL E7Q} & {54} & {BDS3 B3Q} \\
    {22} & {GAL E7X} & {55} & {BDS3 B3X} \\
    {23} & {GAL E8I} & {56} & {GPS L1CI} \\
    {24} & {GAL E8Q} & {57} & {GPS L1CQ} \\
    {25} & {GAL E8X} & {58} & {GPS L1CX} \\
    {26} & {GAL E5I} & {59} & {AUX GPS} \\
    {27} & {GAL E5Q} & {60} & {AUX SBAS} \\
    {28} & {GAL E5X} & {61} & {AUX GAL} \\
    {29} & {GLO L1P} & {62} & {AUX QZS} \\
    {30} & {GLO L2P} & {63} & {AUX BDS} \\
    {31} & {QZS L1CA} & {} & {} \\
    {32} & {QZS L1CI} & {} & {} \\
    \bottomrule
  \end{longtable}
\captionof{table}{Gnss Signals Table}
\end{center}

((* block messages_table *))
((* endblock *))

((* block messages_desc *))
((* endblock *))

\end{document}
